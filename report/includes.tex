%\usepackage[portuguese]{babel} %Portuguese-specific commands
\usepackage[english]{babel} %English-specific commands

\usepackage{hyperref} % used to han­dle cross-ref­er­enc­ing com­mands in LATEX to pro­duce hy­per­text links in the doc­u­ment. 

\usepackage{epstopdf} %con­verts an EPS file to an ‘en­cap­su­lated’ PDF file

\epstopdfsetup{update} % only generate pdf files when eps file is newer

\usepackage{textcomp} %The pack­age sup­ports the Text Com­pan­ion fonts, which pro­vide many text sym­bols (such as baht, bul­let, copy­right, mu­si­cal­note, onequar­ter, sec­tion, and yen), in the TS1 encodings.

\usepackage[hypcap]{caption} %makes \ref point to top of figures and tables

\usepackage{float} %Im­proves the in­ter­face for defin­ing float­ing ob­jects such as fig­ures and ta­bles. In­tro­duces the boxed float, the ruled float and the plaintop float.

\usepackage[nottoc]{tocbibind} %The tocbibind package can be used to add document elements like a bibliography or an index to the Table of Contents. The package is designed to work with the four standard book, report, article and proc classes, and to a limited extent with the ltxdoc class.

\usepackage[utf8]{inputenc} %Package utf8 implements functions and constants to support text encoded in UTF-8. It includes functions to translate between runes and UTF-8 byte sequences.

\usepackage[T1]{fontenc} %The pack­age al­lows the user to se­lect font en­cod­ings, and for each en­cod­ing pro­vides an in­ter­face to ‘font-en­cod­ing-spe­cific’ com­mands for each font. Its most pow­er­ful ef­fect is to en­able hy­phen­ation to op­er­ate on texts con­tain­ing any char­ac­ter in the fon t.The pack­age su­per­sedes t1enc; it is dis­tributed as part of the la­tex dis­tri­bu­tion.

\usepackage{graphicx} %The pack­age builds upon the graph­ics pack­age, pro­vid­ing a key-value in­ter­face for op­tional ar­gu­ments to the \in­clude­graph­ics com­mand. This in­ter­face pro­vides fa­cil­i­ties that go far be­yond what the graph­ics pack­age of­fers on its own.

\usepackage[justification=centering]{caption} %centering caption of a figure

\usepackage{listings} %The pack­age en­ables the user to type­set pro­grams (pro­gram­ming code) within LATEX; the source code is read di­rectly by TEX—no front-end pro­ces­sor is needed. Key­words, com­ments and strings can be type­set us­ing dif­fer­ent styles (de­fault is bold for key­words, italic for com­ments and no spe­cial style for strings). Sup­port for hy­per­ref is pro­vided.

\usepackage{minted} %The pack­age that fa­cil­i­tates ex­pres­sive syn­tax high­light­ing in LATEX us­ing the pow­er­ful Pyg­ments li­brary. The pack­age also pro­vides op­tions to cus­tomize the high­lighted source code out­put us­ing fan­cyvrb.

\usepackage{indentfirst} % indent first paragraph in section

\usepackage{geometry}% http://ctan.org/pkg/geometry
% Increase the text block as far as is needed (in all directions) and leave only a 1cm margin between the text block and the page boundary. This will allow you to fit more on a page, perhaps allowing the figure to be set where you want it to. Note that this option will have no effect if you're using a text block related length (like  \textwidth), since the proportional modification would translate to the figure as well.

\usepackage[svgnames,dvipsnames,table,xcdraw]{xcolor} %The pack­age starts from the ba­sic fa­cil­i­ties of the color pack­age, and pro­vides easy driver-in­de­pen­dent ac­cess to sev­eral kinds of color tints, shades, tones, and mixes of ar­bi­trary col­ors. It al­lows a user to se­lect a doc­u­ment-wide tar­get color model and of­fers com­plete tools for con­ver­sion be­tween eight color mod­els. Ad­di­tion­ally, there is a com­mand for al­ter­nat­ing row col­ors plus re­peated non-aligned ma­te­rial (like hor­i­zon­tal lines) in ta­bles. Colors can be mixed like \color{red!30!green!40!blue}.

\usepackage{sectsty} %A LATEX2ε pack­age to help change the style of any or all of LATEX's sec­tional head­ers in the ar­ti­cle, book, or re­port classes. Ex­am­ples in­clude the ad­di­tion of rules above or be­low a sec­tion ti­tle.

